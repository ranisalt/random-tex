\documentclass[12pt]{article}
\usepackage{sbc/template}
\usepackage{colortbl,hyperref,tabu,titling,xcolor}
\usepackage[brazil]{babel}
\usepackage[utf8]{inputenc}

\title{\bfseries Relatório de Estágio}
\author{Ranieri Schroeder Althoff}

\sloppy

\begin{document}

\begin{center}
    \bfseries\Large
    Universidade Federal de Santa Catarina \\
    Departamento de Informática e de Estatística \\
    Curso de Bacharelado em Ciência da Computação \\
    INE 5444 – Estágio Supervisionado I \\

    \vspace*{5\baselineskip}

    DADOS CADASTRAIS

    \vspace*{5\baselineskip}
\end{center}

\begin{center}
    \bfseries
    \begin{tabular}{| >{\columncolor[gray]{0.95}}l | c |}
        \hline
        Aluno: & Ranieri Schroeder Althoff \\ \hline
        Título do estágio: & Desenvolvimento de softwares e algoritmos \\ \hline
        TCE: & 637480 \\ \hline
        Local do estágio: & Khomp Indústria e Comércio Ltda. \\ \hline
        Data de início do estágio: & 23 nov 2015 \\ \hline
        Data de término do estágio: & 22 nov 2016 \\ \hline
    \end{tabular} \\
\end{center}

\newpage

\begin{center}
    \textbf{\Large\thetitle}\linebreak

    \textbf{\theauthor}\linebreak

    Universidade Federal de Santa Catarina \\
    Departamento de Informática e de Estatística \\
    Curso de Bacharelado em Ciência da Computação \\
    \href{mailto:ranieri.althoff@grad.ufsc.br}{ranieri.althoff@grad.ufsc.br} \linebreak

    INE 5444 - Estágio Supervisionado I
\end{center}

\begin{resumo}
\end{resumo}

\section{Introdução}

O estágio foi realizado na empresa Khomp Indústria e Comércio Ltda., situada na
Rua Joe Collaço, 253, bairro Santa Mônica, Florianópolis. Fundada em 1996,
atua no desenvolvimento de soluções personalizadas \textit{hardware} e
\textit{software} para centrais de comutação telefônica, com o objetivo de
proporcionar equipamentos de alto desempenho, incorporando novas facilidades e
fornecendo produtos de ponta, competitivos e tecnologicamente atualizados. A
empresa hoje conta com mais de 100 empregados e representantes em 4 países.,
além da sede em Florianópolis.

A missão da empresa é: ``desenvolver e fornecer produtos para o mercado de
telecomunicações aliando flexibilidade e eficiência no relacionamento com o
cliente, buscando constantemente, mediante atitudes inovadoras e criativas,
resultados positivos para a nossa Empresa, clientes, acionistas e
colaboradores''.

A visão da empresa é: ``ser uma empresa competitiva e reconhecida
internacionalmente pela inovação e qualidade tecnológica resultante da
capacidade da equipe e de processos ágeis, visando a satisfação dos clientes, a
rentabilidade de nossa empresa e a qualidade de vida das pessoas envolvidas''.

O estágio se deu na área de desenvolvimento de \textit{software}, trabalhando
com aplicações em baixo, médio e alto nível, utilizando algoritmos de
processamento digital de sinais e tecnologias de telecomunicação, como VoIP,
GSM e telefonia analógica.

A escolha da empresa se deu em função de sua localização adequada, próxima da
universidade, facilitando a presença do estagiário, e da flexibilidade do
horário, interessante para alunos de cursos em período integral. Além disso, a
Khomp é uma empresa prestigiada e premiada em suas práticas de estágio,
oferecendo diversos benefícios e oportunidades de aprendizado aos estagiários.
A empresa hoje tem vários estagiários estudantes e empregados formados na UFSC.

A oportunidade do estágio me proporcionou conhecimento sobre uma área que não é
profundamente abrangida no curso, telecomunicações, bem como a experiência de
trabalhar em um \textit{software} de grande porte e com uma equipe vasta e
diferenciada, com especialistas em variadas áreas de conhecimento.

O estágio foi iniciado em dezembro de 2015 e está planejado para se encerrar em
novembro de 2016, por um período de 12 meses, com carga horária de 20 horas
semanais.

\section{Desenvolvimento}

O período inicial do estágio foi de leitura para se obter conhecimento sobre
as ferramentas e métodos utilizados pela empresa e da área de atuação. Foi
fornecido material completo para tal, em artigos, livros e \textit{softwares}
para iniciantes, de forma a acostumar o estagiário ao código que será
desenvolvido no estágio.

Além do próprio \textit{software}, os produtos possuem integração com alguns
\textit{softwares open source}, cujo material está disponível na internet, em
adição ao material fornecido pela empresa.

\subsection{Revisão da literatura}

\subsubsection{PSTN}

A \textbf{rede pública de telefonia comutada}, do inglês
\textit{public switched telephone network} ou PSTN, é uma rede comutada por
circuitos e otimizada para comunicação de voz em tempo real a nível mundial.
Uma ligação telefônica cria um circuito entre dois telefones para estabelecer
comunicação de voz. O circuito criado para uma ligação via PSTN é dedicado,
garantindo qualidade de serviço.

Os primeiros telefones, criados por Alexander Graham Bell e sua companhia, a
\textit{American Bell Telephone Company}, não funcionavam em rede, mas eram
conectados aos pares para uso privado. Um usuário que desejasse ligar para
pessoas diferentes precisava ter um telefone para cada pessoa com quem
desejasse falar. No início, os telefones também eram unidirecionais, e não
existia o conceito do número de telefone.

Não era plausível colocar equipamentos e ligar cabos em cada casa que desejasse
fazer ligações telefônicas, e a \textit{American Bell} desenvolveu o sistema de
comutação, que permitia ligações entre os telefones indiretamente, ou seja, sem
a ligação exclusiva entre os telefones. Dessa forma, uma central telefônica era
encarregada de conectar os telefones e criar o circuito da ligação.

As centrais telefônicas possuiam diversos painéis de comutação, os agrupamentos
destes chamados de \textit{trunks}, formando as primeiras redes telefônicas. As
redes eram conectadas entre centrais, cidades, estados e países, formando então
a PSTN.

\subsubsection{NAPs e roteamento de telefonia}

Os \textit{NAPs}, sigla para \textit{network access point} ou
\textbf{ponto de acesso à rede}, são equipamentos que conectam os dispositivos
usados em uma rede interna à tipos de redes externas, como telefones locais de
um \textit{call center} à PSTN. Suas aplicações são bastante amplas, não se
limitando a essa área.

\subsection{Metodologia}

\subsection{Resultados obtidos}

\section{Conclusões}

%\bibliographystyle{sbc/sbc}
%\bibliography{relatorio}

\end{document}
